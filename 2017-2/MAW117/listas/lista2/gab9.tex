Em janeiro a receita foi
\begin{eqnarray*}
	R(P(0))
	& = & R(55)\\
	& = & -\frac{3\cdot55^3}{5000}+\frac{9\cdot55^2}{50}\\
	& = & 444,675.
\end{eqnarray*}

Junho corresponde a $t=6$, então
\begin{eqnarray*}
	R(P(6))
	& = & R\left(\frac{10}{81}6^3-\frac{10}{3}6^2+\frac{200}{9}6+55\right)\\
	& = & R(95)\\
	& = & -\frac{3\cdot95^3}{5000}+\frac{9\cdot95^2}{50}\\
	& = & 1110,075.
\end{eqnarray*}

Em dezembro:
\begin{eqnarray*}
	R(P(11))
	& = & R\left(\frac{10}{81}11^3-\frac{10}{3}11^2+\frac{200}{9}11+55\right)\\
	& = & R(60,432)\\
	& = & 524,945.
\end{eqnarray*}

