Uma agência de turismo organiza passeios em grupos de até 30 pessoas.
O preço do passeio é de R\$100,00 por pessoa, para grupos de até 20 indivíduos.
Para grupos maiores, a agência aluga um ônibus que tem 30 lugares.
Assim a agência tem interesse em fazer passeios com grupos completos, portanto
ela faz uma promoção: a partir de 20 pessoas, o preço individual do passeio cai em R\$2,00 para cada
pessoa a mais. 
Isto é, um passeio de 20 pessoas sai a R\$100,00 para cada um, um passeio de 21 pessoas sai a R\$98,00 cada um,
para 22 pessoas sai a R\$96,00 cada um etc.
Além disso, um grupo precisa ter no mínimo 5 pessoas.

Escreva as funções $p$ e $C$ (o domínio, contradomínio e expressão de cada uma), onde $p(x)$ é o preço
individual cobrado para grupos de $x$ pessoas e $C(x)$ é o total arrecadado pela agência em um passeio
para um grupo de $x$ pessoas.

