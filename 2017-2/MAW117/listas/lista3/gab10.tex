O argumento dentro da raiz deve ser não negativo.
Além disso, $x$ não pode nem ser 3 nem -3 (denominador não pode ser 0).

A expressão
\[
	\frac{x^2-5x}{x^2-9}=\frac{x(x-5)}{(x-3)(x+3)},
\]
muda de sinal nos pontos $-3,0,3$ e $5$.
Olhando caso a caso:
\begin{enumerate}
	\item $x<-3:$
	Então $x<0,x-5<0, x+3<0$ e $x-3<0$
	Logo a expressão é positiva.

	\item $-3<x<0$.
	Então $x<0,x-5<0,x-3<0$ e $x+3>0$.
	Logo a expressão é negativa.

	\item $0<x<3$.
	Então $x>0,x-3<0,x+3>0,x-5<0$.
	Logo a expressão é positiva.

	\item $3<x<5$.
	Então $x>0,x-3>0,x+3>0$ e $x-5<0$.
	Logo a expressão é negativa.

	\item $x>5$.
	Então $x>0,x-3>0,x+3>0$ e $x-5>0$.
	Logo a expressão é positiva.
\end{enumerate}

O domínio então é
\[
	D=(-\infty,-3[\cup[0,3[\cup[5,+\infty).
\]
