O único ``problema" é a raiz quadrada, logo o domínio de $f$ tem que ser $\R_+=[0,+\infty)$.
Note que
\[
	f(0)=\frac{8}{\sqrt{0}+2}=4
\]
e a medida que tomarmos $x$ maiores, o denominador crescer, logo o valor de $f$ decai até 0 (porém nenhum valor de $x$
alcança 0).
Ou seja, a imagem de $f$ é $]0,4]$.

Tomando
\[
	y=\frac{8}{\sqrt{x}+2},
\]
segue que
\[
	(\sqrt{x}+2)y=8\implies x=\left(\frac{8}{y}-2\right)^2.
\]
Ou seja, $f^{-1}:]0,4]\rightarrow\R_+$ é dada por
\[
	f^{-1}(y)=\left(\frac{8}{y}-2\right)^2.
\]
Em particular, $f^{-1}(1/2)=(8/(1/2)-2)^2=(16-2)^2=14^2=196.$

No caso de $g$, o único ``problema" é caso $x=-4$.
Se $y$ está na imagem de $g$, então
\[
	y=\frac{2x-5}{x+4}\implies yx+4y=2x-5\implies x(y-2)=-5-4y\implies x=\frac{-5-4y}{y-2}
\]
e portanto $y\not=2$.
De fato, é fácil ver que $g(x)=2$ não tem solução.

Então $g:\R\setminus\{-4\}\rightarrow\R\setminus\{2\}$ e $g^{-1}:\R\setminus\{2\}\rightarrow\R\setminus\{-4\}$ dada por
\[
	g^{-1}(y)=\frac{-5-4y}{y-2}.
\]
