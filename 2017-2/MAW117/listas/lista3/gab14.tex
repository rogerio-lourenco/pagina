\begin{enumerate}
	\item Chame $e^x=y$.
	Então $e^{2x}=y^2$ e a equação fica
	\[
		y^2-2y-15=0,
	\]
	que é apenas uma equação do segundo grau.
	Portanto $y$ é um dos valores dados por
	\[
		\frac{2\pm\sqrt{4+4\cdot15}}{2}=1\pm\sqrt{1+15}=1\pm 4.
	\]
	Note que $y=e^x>0$, portanto $y$ pode apenas ser o valor positivo: $y=1+4=5$.
	Então $x=\ln(5).$

	\item \begin{eqnarray*}
		\log_2(x+35)-\log_2x & = & 3\\
		\log_2\left(\frac{x+35}{x}\right) & = & 3\\
		\frac{x+35}{x} & = & 2^3=8\\
		x+35 & = & 8x\\
		35 & = & 7x\\
		x & = & \frac{35}{7}=5.
	\end{eqnarray*}

	\item \begin{eqnarray*}
		\log_6(x-11)+\log_6(x-6) & = & 2\\
		\log_6(x-11)(x-6) & = & 2 \\
		\log_6(x^2-17x+66) & = & 2\\
		x^2-17x+66 & = & 6^2=36 \\
		x^2-17x+30 & = & 0,
	\end{eqnarray*}
	logo
	\[
		x=\frac{17+ 13}{2}=15\mbox{ ou }x=\frac{17-13}{2}=2.
	\]

	\item Chame $y=\ln(x)$.
	Como $\ln(x^2)=2\ln(x)=2y$, a equação fica
	\[
		y^2=2y.
	\]
	Então $y=2\implies \ln(x)=2\implies x=e^2$.
	A outra possibilidade $y=0$ nos dá $x=e^0=1$.

	\item \begin{eqnarray*}
		2\ln(x)-\ln(x+2) & = & 0\\
		\ln\left(\frac{x^2}{x+2}\right) & = & 0\\
		\frac{x^2}{x+2} & = & 1\\
		x^2-x-2 & = & 0\\
		x & = & 2.
	\end{eqnarray*}
	A outra possibilidade ($x=-1$) não é válida, pois temos $\ln(x)$ na equação e o domínio de $\ln$ é $\R^*_+$.

	\item \begin{eqnarray*}
		\ln(x-3)+\ln(x+1) & = & \ln(x+7)\\
		\ln((x-3)(x+1)) & = & \ln(x+7) \\
		(x-3)(x+1) & = & x+7\\
		x^2-2x-3 & = & x+7\\
		x^2-3x-10 & = & 0\\
		x & = & 5.
	\end{eqnarray*}
	A outra possibilidade ($x=2$) não é válida, pois temos $\ln(x-3)$ na equação e o domínio de $\ln$ é $\R^*_+$.
\end{enumerate}
