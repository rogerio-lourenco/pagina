Para calcular $R(2017)$ e $R(2018)$, temos que obter os valores de $\alpha$ e $\beta$.
Para isso, tome dois anos em que temos a receita, por exemplo, 2010 e 2011.
Então
\[
	\left\{
		\begin{array}{l}
			R(2010)=2010\alpha+\beta=1;\\
			R(2011)=2011\alpha+\beta=1,23.
		\end{array}
	\right.
\]
Portanto, subtraindo a primeira equação da segunda, temos que
\[
	(2011\alpha+\beta)-(2010\alpha+\beta)=1,23-1
	\implies \alpha=0,23.
\]
Tendo o valor de $\alpha$, temos
\[
	2010\alpha+\beta=1\implies \beta=1-2010\cdot0,23=1-462,3=-461,3
\]
Portanto
\[
	R(2017)=2017\cdot0,23-461,3=2,61
\]
e
\[
	R(2018)=2018\cdot0,23-461,3=2,84.
\]

Agora, para termos $R(x)\geq 4$, temos que ter
\[
	\alpha x+\beta\geq 4\implies x\geq \frac{4-\beta}{\alpha}=\frac{4+461,3}{0,23}
	\approx 2023.
\]
