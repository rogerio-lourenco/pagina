\begin{enumerate}
	\item $f(0)=0^2+1=1$;
	\item $g(4)=\sqrt{|4|}=\sqrt{4}=2;$
	\item Como $0<1$, $h(0)=0-1=-1;$
	\item Como $3/2\geq 1$, $h(3/2)=3/2+1=5/2;$
	\item $(g\circ f)(x)=g(f(x))=\sqrt{|x^2+1|}=\sqrt{x^2+1}$, já que $x^2+1\geq 
1,\forall x\in\R;$
	\item $(f\circ g)(x)=\left(\sqrt{|x|}\right)^2+1=|x|+1;$
	\item Como $(g\circ f)(x)=\sqrt{x^2+1}\geq 1,\forall x\in\R$, temos que
	\[
		(h\circ g\circ f)(x)=(g\circ f)(x)+1=\sqrt{x^2+1}+1.
	\]
	\item Novamente, como $(f\circ g)(x)=|x|+1\geq 1,\forall x\in\R$,
	\[
		(h\circ f\circ g)(x)=(|x|+1)+1=|x|+2.
	\]
\end{enumerate}
