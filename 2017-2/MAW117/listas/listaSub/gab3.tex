Se $x_1$ e $x_2$ são as raízes, então
\begin{eqnarray*}
	a(x-x_1)(x-x_2)
	& = & a(x^2+x(-x_1-x_2)+x_1x_2)\\
	& = & ax^2+xa(-x_1-x_2)+ax_1x_2\\
	& = & ax^2+bx+c,
\end{eqnarray*}
Logo
\[
	a(-x_1-x_2)=b\mbox{ e }ax_1x_2=c.
\]

\begin{enumerate}
	\item Pela observação acima,
	\[
		x_1+x_2=\frac{-b}{a}.
	\]
	\item Temos também
	\[
		x_1x_2=\frac{c}{a}.
	\]
	\item Se $x_1=-x_2$, então
	\[
		-2m=\frac{-b}{a}=x_1+x_2=0\implies m=0.
	\]
\end{enumerate}
