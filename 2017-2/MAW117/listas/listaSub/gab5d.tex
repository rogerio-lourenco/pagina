Temos que resolver
\[
	\sqrt{1-3x}-\sqrt{5+x}>1\implies \sqrt{1-3x}>1+\sqrt{5+x}.
\]
Como as raízes tem que ser reais, os termos dentro das raízes têm que ser não-negativos.
Portanto
\[
	1-3x\geq 0\implies x\leq \frac{1}{3}
\]
e
\[
	5+x\geq 0\implies x\geq -5.
\]
Ou seja, para começar, temos que considerar apenas valores no intervalo $\left[-5,\frac{1}{3}\right]$.

Agora, como o termo do lado esquerdo da inequação é positivo (já que é maior do que 1), podemos elevar ao quadrado ambos os lados:
\[
	1-3x>1+(5+x)+2\sqrt{5+x}\implies -5-4x>2\sqrt{5+x}\geq 0.
\]
Como a expressão acima tem que ser não-negativa, temos que $-5-4x>0\implies x<\frac{-5}{4}$, o que permite restringir o intervalo
inicial para
\[
	\left[-5;\frac{-5}{4}\right].
\]

Sendo ambos os lados não-negativos, podemos elevar os dois lados ao quadrado:
\[
	25+40x+16x^2>4x+20\implies 16x^2+36x+5>0.
\]
A inequação acima é válida para
\[
	x\in\left(-\infty;\frac{-9-\sqrt{61}}{8}\right)\cup\left(\frac{-9+\sqrt{61}}{8};+\infty\right).
\]
Lembrando que já temos uma restrição para os possíveis valores de $x$, temos que a inequação inicial é valida para
\[
	x\in\left[-5;\frac{-9-\sqrt{61}}{8}\right).
\]
